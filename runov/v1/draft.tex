% Методы балансировки высоконагруженных систем

% Описание проблемы. Что такое методы балансировки и зачем в высоконагруженных
% системах они нужны? Что такое балансировка и зачем нужна?
% Что такое балансировщик? Как работает?

\section{Введение}

В современном мире активно развиваются интернет-компании. По мере роста, им
приходится иметь дело с постоянно возрастающими количеством пользователей и
объемом обрабатываемых данных, и в какой-то момент возникает потребность в
увеличении количества серверов, а затем -- в эффективном распределении нагрузки
между ними. Для этого используются так называемые \textbf{балансировщики
нагрузки}. Балансировщик нагрузки выступает в роли <<регулировщика>>, стоящего
перед серверами и направляющего запросы клиентов на все серверы, способные
выполнить эти запросы таким образом, чтобы максимально увеличить скорость и
загрузку мощностей и не допустить перегрузки одного сервера, что может снизить
производительность. Если один из серверов выходит из строя, балансировщик
нагрузки перенаправляет трафик на оставшиеся работающие серверы. Для выполнения
своих функций балансировщик нагрузки использует различные методы балансировки.
\cite{nginxlb}

% Цель работы

Целью данной работы является классификация методов балансировки
высоконагруженных систем.

% Что нужно для достижения цели

Для достижения поставленной цели решаются следующие задачи:
% \begin{itemize}
% \item
    Рассматриваются известные подходы к балансировке высоконагруженных систем;
% \item
    Описываются методы балансировки высоконагруженных систем, относящиеся к каждому из подходов;
% \item
    Предлагаются и обосновываются критерии оценки качества описанных методов;
% \item
    Сравниваются методы по предложенным критериям оценки;
% \item
    Выделяются методы, показывающие лучшие результаты по одному или нескольким критериям.
% \end{itemize}

% Какие существуют алгоритмы балансировки

В данной работе будут рассмотрены следующие методы балансировки нагрузки:
\texttt{Source IP-Hash}, \texttt{URL-Hash}, \texttt{Fixed Weighting}.

\section{Аналитический раздел}

Алгоритмы балансировки нагрузки принято разделять на две категории: \textbf{статические} и \textbf{динамические}.

% Статические алгоритмы подходят для систем с небольшими колебаниями нагрузки. В статических алгоритмах трафик распределяется равномерно между серверами. Эти алгоритмы требуют предварительных сведений о ресурсах системы. Производительность процессоров определяется в начале выполнения, поэтому решение о перераспределении нагрузки не зависит от текущего состояния системы. Тем не менее, статические алгоритмы балансировки нагрузки имеют недостаток, заключающийся в том, что задачи назначаются процессору или машинам 

В \textbf{статических} алгоритмах балансировки нагрузки решения о распределении нагрузки между узлами системы принимаются в начале выполнения, они не зависят от текущего состояния системы. Цель статической балансировки нагрузки -- снизить общее время выполнения синхронной программы, минимизируя коммуникационные задержки. \cite{lbiccar}

В \textbf{динамических} алгоритмах балансировки нагрузки решения о распределении нагрузки принимаются на основе текущего состояния системы, предварительных сведений о системе для этого не требуется. Главное преимущество динамической балансировки нагрузки заключается в том, что отказ одного из узлов не приведёт к остановке системы, а лишь повлияет на ее производительность. \cite{lbiccar}

\subsection{{Source IP-Hash}}

\texttt{Source IP-Hash} -- динамический алгоритм балансировки нагрузки, который объединяет исходный и целевой IP-адреса клиента и сервера для создания уникального хеш-ключа. Ключ используется для привязки клиента к определенному серверу. Поскольку ключ может быть повторно сгенерирован, если сеанс прерван, запрос клиента направляется на тот же сервер, который он использовал ранее. Это полезно, если важно, чтобы клиент подключился к сеансу, который все еще активен после отключения. \cite{sdn5g}

\subsection{{URL-Hash}}

\texttt{URL-Hash} -- динамический алгоритм балансировки нагрузки, который так же, как и \texttt{Source IP-Hash} использует хеш-ключ для маршрутизации. Отличие состоит лишь в том, что для генерации хеш-ключа используется {URL}-адрес интернет-ресурса, к которому происходит обращение. \cite{wwwc}

\subsection{{Fixed Weighting}}

\texttt{Fixed Weighting} -- динамический алгоритм балансировки нагрузки, при котором администратор назначает вес каждому серверу приложений на основе критериев по их выбору, чтобы продемонстрировать способность серверов приложений обрабатывать трафик. Сервер приложений с наибольшим весом получит весь трафик. Если сервер приложений с наибольшим весом выходит из строя, весь трафик будет направлен на следующий сервер приложений с наибольшим весом. \cite{sdn5g}
