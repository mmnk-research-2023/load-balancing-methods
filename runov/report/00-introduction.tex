\phantomsection\section*{ВВЕДЕНИЕ}\addcontentsline{toc}{section}{ВВЕДЕНИЕ}

Для обеспечения бесперебойной работы своих интернет-ресурсов, компаниям необходимо обеспечить бесперебойною работу своих серверов.
Сервера могут выходить из строя по многим причинам: аппаратные сбои, проблемы с электропитанием, человеческие ошибки, перегрузка; в связи с чем, у компаний возникает естесственная потребность --- минимизировать вероятность выхода из строя своих серверов.
От аппаратных сбоев и человеческого фактора защититься бывает крайне сложно, но если грамотно распределять между серверами поступающий объём данных, можно уменьшить вероятность выхода серверов из строя в следствие перегрузки.
% Для достижения эффективного распределения поступающего объёма данных между серверами существуют специальные устройства, называемые <<балансировщиками нагрузки>>, а также методы балансировки нагрузки, согласно которым эти устройства работают.
% Для достижения эффективного распределения поступающего объёма данных между серверами, можно на один из серверов возложить ответственность за распределение входящего трафика между другими серверами системы.
% В качестве такого устройства может выступать сервер, выполняющий функции распределения входящего трафика между другими серверами сис-темы.

Этим и занимаются так называемые <<балансировщики нагрузки>>.
Балансировщик нагрузки выполняет роль <<регулировщика>>, стоящего перед серверами и направляющего запросы клиентов на все серверы, способные выполнить эти запросы таким образом, чтобы максимально увеличить скорость и загрузку мощностей и не допустить перегрузки одного сервера, что может привести к снижению производительности.
Если один из серверов выходит из строя, балансировщик нагрузки перенаправляет трафик на оставшиеся работающие серверы.
При добавлении нового сервера в группу серверов балансировщик нагрузки автоматически начинает направлять на него запросы.~\cite{nginx}

Сервер, на который следует направить клиентский запрос, балансировщики нагрузки выбирает в соответствии с различными алгоритмами и методами балансировки нагрузки.

Целью данной работы является анализ и классификация методов балансировки нагрузки высоконагруженных систем.

Для достижения поставленной цели необходимо решить следующие задачи:
\begin{itemize}
    \item рассмотреть основные подходы к балансировке нагрузки;
    \item описать методы балансировки нагрузки, относящиеся к одному из подходов;
    \item предложить и обосновать критерии качества описанных методов;
    \item сравнить методы по предложенным критериям оценки;
    \item выделить методы, показывающие лучшие результаты по одному или нескольким критериям.
\end{itemize}
