\section{Анализ предметной области}

TODO

\subsection{Актуальность проблемы}

TODO

\subsection{Области применения балансировки нагрузки}

TODO

\section{Описание существующих решений}

TODO

\subsection{Статическая балансировка}

TODO

\subsection{Динамическая балансировка}

TODO

% \begin{figure}[H]
% 	\centering
% 	\includegraphics[scale=0.5]{img/lorem.png}
% 	\caption{Lorem Ipsum}
% 	\label{fig:fig}
% \end{figure}

\subsubsection{Методы на основе хеширования}

Методы балансировки нагрузки на основе хеширования работают по общему принципу:
\begin{enumerate}
    \item Из пришедшего запроса выбрать информацию (например, IP-адрес или URL-адрес), которая считается ключом хеш-функции в рамках данного алгоритма.
    \item На основе ключа вычислить значение хеш-функции, которое соответствует идентификатору узла, на который следует перенаправить запрос для его обработки.
    \item Перенаправить запрос на узел, чей идентификатор был вычислен ранее.
\end{enumerate}

\subsubsection*{Хеширование на основе IP-адреса}

Алгоритм балансировки нагрузки <<Хеширование на основе IP-адреса>> работает по общему принципу методов балансировки нагрузки на основе хеширования.
Ключом хеш-функции в данном алгоритме считается IP-адрес источника запроса.
Запросы, имеющие один и тот же IP-адрес, будут обслужены одним и тем же узлом.
То есть, если имеются запросы $r_1$, $r_2$, узлы $l_1$, $l_2$, моменты времени $t_1$, $t_2$ и функция расписания $s$, то, в соответствии с данным алгоритмом, будет выполнено следующее:
\begin{equation}
    (\forall t_1, t_2) \left(\begin{cases}
        s(l_1, t_1) = r_1, \\
        s(l_2, t_2) = r_2, \\
        r_1.ip\_address = r_2.ip\_address.
    \end{cases} \Rightarrow l_1 = l_2 \right)
\end{equation}

Особенности алгоритма <<Хеширование на основе IP-адреса>>:
\begin{itemize}
    \item алгоритм гарантирует, что все запросы от одного и того же пользователя направляются на тот же сервер;
    \item алгоритм предсказуем;
    \item добавление новых серверов потребует лишь изменения хеш-функции для корректной работы алгоритма;
    \item неравномерная нагрузка на узел, если запросы начинают приходить из сети, использующей NAT, с большим количеством пользователей.
\end{itemize}

\subsubsection*{Хеширование на основе URL-адреса}

Алгоритм балансировки нагрузки <<Хеширование на основе URL-адреса>> работает по общему принципу методов балансировки нагрузки на основе хеширования.
Ключом хеш-функции в данном алгоритме считается URL-адрес, к которому обращается источник запроса.
Запросы к одному и тому же URL-адресу, будут обслужены одним и тем же узлом.
То есть, если имеются запросы $r_1$, $r_2$, узлы $l_1$, $l_2$, моменты времени $t_1$, $t_2$ и функция расписания $s$, то, в соответствии с данным алгоритмом, будет выполнено следующее:
\begin{equation}
    (\forall t_1, t_2) \left(\begin{cases}
        s(l_1, t_1) = r_1, \\
        s(l_2, t_2) = r_2, \\
        r_1.url\_address = r_2.url\_address.
    \end{cases} \Rightarrow l_1 = l_2 \right)
\end{equation}

Особенности алгоритма <<Хеширование на основе URL-адреса>>:
\begin{itemize}
    \item алгоритм гарантирует, что все запросы к одному и тому же URL направляются на тот же сервер;
    \item алгоритм предсказуем;
    \item изменение структуры URL может потребовать перенастройки балансировщика;
    \item популярные URL могут создавать неравномерную нагрузку на узлы.
\end{itemize}

\subsubsection{Метод фиксированных весов}

В методе фиксированных весов, администратор назначает каждому узлу вес, после чего все запросы будут приходить на узел с максимальным весом.
Если узел перестаёт справляться с нагрузкой, запросы начинают перенаправляться на узел, с весом меньше.

Шаги инициализации алгоритма:
\begin{enumerate}
    \item N = количество узлов
    \item nodes = массив узлов
    \item weights = массив весов таких, что weights[i] --- вес узла i, назначенный администратором
    \item i = 1
    \item Пока i <= N:
        \begin{itemize}
            \item nodes[i].weight = weights[i]
            \item i = i + 1
        \end{itemize}
\end{enumerate}

Шаги работы алгоритма:
\begin{enumerate}
    \item w = max(weights)
    \item request\_sent\_flag = 0
    \item Пока request\_sent\_flag = 0 и w > 0:
        \begin{itemize}
            \item node = узел с весом w
            \item Если узел node работоспособен, перенаправить запрос узлу node, request\_sent\_flag = 1
            \item Иначе, w = w - 1
        \end{itemize}
\end{enumerate}

Таким образом, если имеется запрос $r$, узел $l$, момент времени $t$, функция расписания $s$, и выполняется равенство $s(l, t) = r$ то, $l$ --- узел с наибольшим весом, доступный в момент времени $t$.

Особенности метода фиксированных весов:
\begin{itemize}
    \item алгоритм гарантирует, что все запросы будут направляться на доступный в текущий момент времени узел с максимальным весом;
    \item алгоритм предсказуем;
    \item веса узлов назначаются вручную;
    \item вес узла не меняется в процессе работы.
\end{itemize}

\section{Классификация существующих решений}

TODO

\subsection{Иерархия методов}

TODO

\subsection{Оценка и сравнение}

TODO

\subsection{Вывод}

TODO
