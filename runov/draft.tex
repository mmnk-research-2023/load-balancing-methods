% Методы балансировки высоконагруженных систем

% Описание проблемы. Что такое методы балансировки и зачем в высоконагруженных
% системах они нужны? Что такое балансировка и зачем нужна?
% Что такое балансировщик? Как работает?

В современном мире активно развиваются интернет-компании. По мере роста, им
приходится иметь дело с постоянно возрастающими количеством пользователей и
объёмом обрабатываемых данных, и в какой-то момент возникает потребность в
увеличении количества серверов, а затем -- в эффективном распределении нагрузки
между ними. Для этого используются так называемые \textbf{балансировщики
нагрузки}. Балансировщик нагрузки выступает в роли <<регулировщика>>, стоящего
перед серверами и направляющего запросы клиентов на все серверы, способные
выполнить эти запросы таким образом, чтобы максимально увеличить скорость и
загрузку мощностей и не допустить перегрузки одного сервера, что может снизить
производительность. Если один из серверов выходит из строя, балансировщик
нагрузки перенаправляет трафик на оставшиеся работающие серверы. Для выполнения
своих функций балансировщик нагрузки использует различные методы балансировки.
\cite{nginxlb}

% Цель работы

Целью данной работы является классификация методов балансировки
высоконагруженных систем.

% Что нужно для достижения цели

Для достижения поставленной цели решаются следующие задачи:
% \begin{itemize}
% \item
    Рассматриваются известные подходы к балансировке высоконагруженных систем;
% \item
    Описываются методы балансировки высоконагруженных систем, относящиеся к каждому из подходов;
% \item
    Предлагаются и обосновываются критерии оценки качества описанных методов;
% \item
    Сравниваются методы по предложенным критериям оценки;
% \item
    Выделяются методы, показывающие лучшие результаты по одному или нескольким критериям.
% \end{itemize}

% Какие существуют алгоритмы балансировки

В данной работе будут рассмотрены следующие методы балансировки нагрузки:
\texttt{Source IP-Hash}, \texttt{URL-Hash}, \texttt{Fixed Weights}.
