\chapter*{ВВЕДЕНИЕ}
\addcontentsline{toc}{chapter}{ВВЕДЕНИЕ}

В современном обществе практически все общение и взаимодействие с приложениями осуществляются через интернет. Сетевые приложения востребованы как обычными пользователями, так и крупными корпорациями, проводящими сложные вычисления и обмен данными~\cite{webact}.

За последние 5 лет количество пользователей в интернете выросло на 30 процентов~\cite{statistics}, что привело к резкому увеличению нагрузки на многие системы. Обработка выросшей нагрузки требует добавления в систему новых вычислительных узлов, для эффективной работы которых, необходимо осуществлять балансировку нагрузки~\cite{strategies, com_analysis, gaud, part_algos}.

Целью данной работы является описание методов балансировки нагрузки в высоконагруженных системах.

Балансировка нагрузки --- это механизм приблизительного выравнивания рабочей нагрузки между всеми узлами системы~\cite{anal}.

Для достижения поставленной цели необходимо выполнить следующие
задачи:
\begin{itemize}
	\item описать принципы балансировки нагрузки;
	\item классифицировать методы балансировки нагрузки;
	\item описать алгоритмы, используемые для балансировки нагрузки.
\end{itemize}

