\chapter*{ЗАКЛЮЧЕНИЕ}
\addcontentsline{toc}{chapter}{ЗАКЛЮЧЕНИЕ}

% В результате проделанной научно-исследовательской работы, была описана предметная область балансировки нагрузки. 

% Было описано, что задача балансировки состоит в минимизации времени обслуживания запросов, а основными параметрами алгоритмов являются: отказоустойчивость, точность прогнозирования и время обработки запроса.
% Были рассмотрены такие алгоритмы статической балансировки, как Round Robin и Weighted Round Robin.
% Также были приведены основные различия между статическими и динамическими алгоритмами, в качестве примера динамических алгоритмов был рассмотрен алгоритм Dynamic Round Robin.

% В результате, была достигнута цель научно-исследовательской работы, а именно, описаны методы балансировки нагрузки в высоконагруженных системах.

В процессе выполнения данной научно-исследовательской работы была описана предметная область балансировки нагрузки, формализована задача балансировки нагрузки, а также были:
\begin{itemize}
    \item описаны основные подходы к решению задачи балансировки нагрузки;
    \item сформулированы критерии сравнения методов решения задачи балансировки нагрузки;
    \item классифицированы методы решения задачи балансировки нагрузки.
\end{itemize}

Таким образом, все задачи для достижения цели данной научно-исследовательской работы были решены, и цель работы была достигнута.
