\chapter*{ВВЕДЕНИЕ}
\addcontentsline{toc}{chapter}{ВВЕДЕНИЕ}
\begin{comment}
Современные приложения должны обрабатывать запросы миллионов пользователей одновременно, а также возвращать каждому пользователю правильный текст, видео, изображения и другие данные. Для обработки таких значительных объемов трафика в большинстве приложений используется множество серверов ресурсов с дублированием данных между ними.
Балансировка нагрузки направляет и контролирует интернет-трафик между серверами приложений и их посетителями или клиентами. В результате повышается доступность, масштабируемость, безопасность и производительность приложения. Алгоритм балансировки нагрузки – это набор правил, которым следует балансировщик нагрузки для определения наилучшего сервера для каждого из различных клиентских запросов.\cite{balance}

Системы, работающие с высокой нагрузкой, такие как веб-серверы, облачные платформы, базы данных и другие, требуют эффективной балансировки нагрузки для обеспечения стабильной и надежной работы.


Цель работы - исследование методов балансировки высоконагруженных систем.

В вычислительной технике балансировка нагрузки улучшает распределение рабочих нагрузок по нескольким вычислительным ресурсам: компьютерам, компьютерным кластерам, сетевым подключениям, центральным процессорам или дисковым устройствам. Балансировка нагрузки призвана оптимизировать использование ресурсов, максимально увеличить пропускную способность, минимизировать время отклика и избежать перегрузки отдельных ресурсов. Применение вместо одного компонента нескольких компонентов с балансировкой может повысить надёжность и доступность благодаря получившемуся запасу мощностей. Балансировка нагрузки обычно подразумевает использование специального ПО или оборудования вроде многоуровневого коммутатора или DNS-сервера.\cite{3}
\end{comment}

На заре развития компьютеры (или ЭВМ, электронно-вычислительные машины) были очень дорогим и штучным инструментом, позволить который могли себе только наиболее крупные институты и предприятия.
Вычислительные ресурсы приходилось экономить всеми возможными способами.
Первые разработчики писали код в режиме «офлайн» и передавали их оператору ЭВМ, который последовательно вводил программы в машину и производил расчеты.
В начале 1960-х годов зародилась концепция разделения времени – распределение вычислительных ресурсов между несколькими пользователями: пока один вводит данные, машина занимается расчетами других \cite{2}.
С увеличением масштабов компьютерных систем — когда они начали состоять из сотен единиц — и ростом мощности, механизмы разделения времени перестали быть актуальными.
Понадобились средства, которые бы управляли нагрузкой на множестве компьютеров сразу \cite{1}. 

Балансировка нагрузки - это механизм, который позволяет перемещать задания с одного компьютерана другой в рамках распределенной системы (это процесс приблизительного выравнивания рабочей нагрузки между всеми узлами распределенной системы).
Это ускоряет обслуживание заданий, например, сводит к минимуму время отклика на задание и повышает эффективность использования ресурсов.
Некоторые из основных целей алгоритма балансировки нагрузки, как указано в [8], заключаются в следующем: (1) добиться большего общего улучшения производительности системы при разумных затратах, например, сократить время отклика задачи при сохранении приемлемых задержек; (2) одинаково относиться ко всем заданиям в системе независимо от их происхождения; (3) обладать отказоустойчивостью:
выносливостью производительности при частичном сбое в системе; (4) иметь возможность модифицировать себя в соответствии с любыми изменениями или расширяться в конфигурации распределенной системы; и (5) поддерживать стабильность системы: способность учитывать чрезвычайные ситуации, такие как внезапный всплеск поступлений, чтобы производительность системы не ухудшалась сверх определенного порога, одновременно предотвращая, чтобы узлы распределенной системы тратили слишком много времени на передачу заданий между собой вместо выполнения эти рабочие места.
\cite{4}




Это лучше для аналита:
Различные исследования, например, [2]- [18], показали, что балансировка нагрузки между узлами распределенной системы значительно повышает производительность системы и увеличивает использование ресурсов. 
Согласно [21], балансировка нагрузки позволяет еще больше снизить среднее и стандартное отклонение времени отклика задачи больше, чем при распределении нагрузки.\cite{4}.


\begin{comment}
В поисках путей решения проблемы оптимизации использования вычислительных ресурсов командой инженеров IBM был предложен новый подход – в рамках одной ЭВМ предоставить каждому пользователю виртуальную машину со своей ОС.
Виртуализация обладала существенными преимуществами над концепцией разделения времени:
\begin{itemize}
	\item Увеличенные надежность и безопасность за счет изоляции пользователей.
	\item Запуск любых приложений (не только приспособленных к концепции разделения времени) за счет симуляции отдельного компьютера для каждого пользователя.
	\item Увеличенная производительность за счет использования легковесных гостевых ОС.
\end{itemize}
\cite{2}


Как мы говорили в одном из предыдущих материалов, в эпоху мейнфреймов — больших универсальных серверов — компании с помощью функции time-sharing делили вычислительные ресурсы одной системы среди нескольких пользователей. На протяжении десятков лет такой подход работал исправно.

С увеличением масштабов сетей — когда они начали состоять из сотен компьютеров — и ростом мощности серверов механизмы разделения времени перестали быть актуальными. Понадобились средства, которые бы управляли нагрузкой на множестве компьютеров сразу. \cite{1}
\end{comment}