\begin{definitions}
	%\definition{Пул ресурсов}{это логическое объединение серверов управления или серверов шлюзов, которые распределяют между собой рабочие нагрузки и принимают на себя рабочие нагрузки в случае сбоя одного из членов.\cite{resourcepool}}
	%\definition{Балансировка нагрузки}{это метод равномерного распределения сетевого трафика по пулу ресурсов, поддерживающих приложение.\cite{balance}}
	\definition{Балансировщик нагрузки}{это устройство, которое находится между пользователем и группой серверов и действует как невидимый посредник, обеспечивая одинаковое использование всех серверов ресурсов.\cite{balance}}
	\definition{Время ответа}{это общее время, затрачиваемое сервером на обработку входящих запросов и отправку ответа.\cite{balance}}
	\definition{Вес}{вероятность, с которой балансировщик нагрузки в следующий раз выберет этот узел \cite{weightedroundrobin}.}
	\definition{Вычислительный узел (узел)}{устройство, выполняющее основную логику обработки запроса\cite{uzel}}
	Распределенная система - это вычислительная среда, в которой различные компоненты распределены между несколькими компьютерами (или другими вычислительными устройствами)
\end{definitions}