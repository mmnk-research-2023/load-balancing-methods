\chapter{Анализ предметной области}

Балансировка нагрузки --- это механизм, который позволяет перемещать задания с одного компьютерана другой в рамках распределенной системы (это процесс приблизительного выравнивания рабочей нагрузки между всеми узлами распределенной системы).
Это ускоряет обслуживание заданий, например, сводит к минимуму время отклика на задание и повышает эффективность использования ресурсов.
Некоторые из основных целей алгоритма балансировки нагрузки, как указано в [8], заключаются в следующем: (1) добиться большего общего улучшения производительности системы при разумных затратах, например, сократить время отклика задачи при сохранении приемлемых задержек; (2) одинаково относиться ко всем заданиям в системе независимо от их происхождения; (3) обладать отказоустойчивостью:
выносливостью производительности при частичном сбое в системе; (4) иметь возможность модифицировать себя в соответствии с любыми изменениями или расширяться в конфигурации распределенной системы; и (5) поддерживать стабильность системы: способность учитывать чрезвычайные ситуации, такие как внезапный всплеск поступлений, чтобы производительность системы не ухудшалась сверх определенного порога, одновременно предотвращая, чтобы узлы распределенной системы тратили слишком много времени на передачу заданий между собой вместо выполнения эти рабочие места. \cite{4}	

Cледует еще четко понимать, о чем идет речь: о распределении или балансировке нагрузки. Несмотря на свою схожесть, эти понятия нельзя назвать взаимозаменяемыми. Так, распределение нагрузки предполагает ее равномерное разделение между серверами. А вот балансировка – это уже ее частный случай, учитывающий ряд факторов, подверженных изменению. \cite{4}

Алгоритмы балансировки разделяют на статические и динамические \cite{uzel}.
\section{Статическая балансировка}
Статическая балансировка – это метод распределения нагрузки на узлы, основанный на заранее определенных параметрах.
Основная цель статической балансировки состоит в том, чтобы равномерно распределить трафик между несколькими узлами, чтобы каждый узел получал приблизительно одинаковое количество нагрузки.

При использовании статической балансировки узлы могут быть настроены по разному:
\begin{enumerate}[label=---]
	\item По равномерному распределению нагрузки: каждый узел получает аналогичное количество запросов.
	\item По приоритету: выделенные узлы имеют больший приоритет и получают больше запросов.
	\item По типу запроса: различные типы запросов могут быть отправлены на разные узлы для оптимизации обработки.
\end{enumerate}

Процесс работы статической балансировки выглядит следующим образом:
\begin{enumerate}[label=---]
	\item Балансировщик нагрузки получает запрос от клиента.
	\item Балансировщик принимает решение о том, на какой сервер направить запрос, основываясь на предварительных настройках.
	\item Балансировщик перенаправляет запрос на выбранный сервер.
	\item Сервер обрабатывает запрос и отправляет ответ клиенту через балансировщик.
\end{enumerate}

Основные принципы статической балансировки:
\begin{enumerate}[label=---]
	\item Предварительная настройка: перед запуском системы настраивается конфигурация балансировщика нагрузки. В этой конфигурации определяются узлы, которые будут использоваться, а также правила распределения нагрузки.
	\item Равномерное распределение: статический балансировщик нагрузки следит за текущим состоянием узлов и равномерно распределяет нагрузку между ними в соответствии с заданными правилами.
	\item Отказоустойчивость: статический балансировщик нагрузки способен обнаруживать отказы узлов и исключать их из распределения нагрузки. Это позволяет обеспечить непрерывную работу системы даже при отказе одного или нескольких узлов.
\end{enumerate}

Преимущества статической балансировки:
\begin{enumerate}[label=---]
	\item Простота настройки: статическая балансировка не требует сложной конфигурации и может быть быстро настроена.
	\item Эффективность: статическая балансировка обеспечивает равномерное распределение нагрузки между всеми узлами, что позволяет повысить производительность и отзывчивость системы.
	\item Надежность: отказоустойчивость статического балансировщика нагрузки позволяет обеспечить непрерывную работу системы в случае отказа одного или нескольких узлов.
\end{enumerate}

Однако статическая балансировка также имеет свои недостатки:
\begin{enumerate}[label=---]
	\item Отсутствие гибкости: при статической балансировке узлы настраиваются заранее и не могут мгновенно реагировать на изменение нагрузки или состояния системы.
	\item Одиночная точка отказа: если балансировщик отказывает, вся система может быть недоступна.
\end{enumerate}

В целом, статическая балансировка является эффективным методом распределения нагрузки, но ее ограничения могут стать проблемой в высоконагруженных или динамических средах. В таких случаях могут быть предпочтительнее динамические методы балансировки, которые позволяют автоматически адаптироваться к изменениям нагрузки.
Статическая балансировка является эффективным и надежным методом равномерного распределения нагрузки на сервера. Она позволяет повысить производительность, обеспечить отказоустойчивость системы, экономить ресурсы и легко масштабировать систему по мере роста нагрузки.

Статическая балансировка является одним из методов равномерного распределения нагрузки на сервера и имеет некоторые преимущества по сравнению с динамической балансировкой.
\cite{6}

\section{Динамическая балансировка}
Динамические алгоритмы осуществляют мониторинг состояния каждого
из узлов и выбирают «наилучший», в рассматриваемый момент времени, из них \cite{webmanage}. К динамическим алгоритмам относятся:
\begin{itemize}
	\item Least Connections
	\item Weighted Least Connections
	\item Least Time
\end{itemize}
\subsection{Least Connections}
Алгоритм Least Connections распределяет нагрузку между узлами в зависимости от количества активных соединений, обслуживаемых каждым узлом. Узел с наименьшим числом соединений будет обрабатывать следующий запрос, и узлы с большим числом соединений будет перераспределять свою нагрузку на узлы с меньшей загрузкой \cite{leastconnection}. 
% %Этот метод планирования включает в себя один из алгоритмов динамического планирования, так как он требует динамического вычисления активных соединений для каждого реального сервера.
%На~рисунке~\ref{img:leastconnections} проиллюстрирована работа метода наименьшего количества подключений  

% Этот метод планирования хорошо подходит для сглаживания распределения, когда поступает много запросов. Например, если есть два службы HTTP, такие как HTTP-1 (с 3 активными HTTP-транзакциями) и HTTP-2 (с 1 активной HTTP-транзакцией), то следующий запрос будет отправлен на службу HTTP-2, потому что HTTP-1 имеет больше активных транзакций, чем HTTP-2.

%\includeimage
%{leastconnections} % Имя файла без расширения (файл должен быть расположен в директории inc/img/)
%{f} % Обтекание (без обтекания)
%{h} % Положение рисунка (см. figure из пакета float)
%{\textwidth} % Ширина рисунка
%{Метод наименьшего количества подключений} % Подпись рисунка

\subsection{Weighted Least Connections}

Данный алгоритм комбинирует принципы алгоритмов Least Connections и Weighted Round Robin \cite{weightedroundrobin}. %\cite{mainsource}. 
Он учитывает как веса узлов, так и количество активных соединений. 
% Серверы с более высоким значением веса будут получать больший процент активных подключений в любой момент времени. 
%Вес сервера по умолчанию равен единице, и администратор IPVS или программа мониторинга могут назначить любой вес реальному серверу. 
Новое сетевое подключение предоставляется узлу, который имеет минимальное отношение количества текущих активных подключений к его весу \cite{mainsource}.

%В алгоритме планирования наименьшего взвешенного соединения (WLC) каждому серверу могут быть присвоены различные весовые коэффициенты производительности. Взвешенный алгоритм планирования наименьшего количества подключений делает с алгоритмом наименьшего количества подключений то же, что взвешенный алгоритм циклического перебора делает с алгоритмом циклического перебора. То есть, он вводит "вес", который основан на спецификациях каждого сервера.

\subsection{Least Time}
Алгоритм Least Time сочетает время отклика узла и активные соединения для определения лучшего узла \cite{balance}. 

Основные принципы метода наименьшего времени ответа включают:

\begin{itemize}
	\item Измерение времени ответа: Для эффективной работы этого метода необходимо непрерывно измерять время ответа от узла. Это может быть выполнено с помощью мониторинга, сбора статистики или других средств измерения производительности.
	
	\item Выбор узла с наименьшим временем ответа: Когда клиент отправляет запрос, система выбирает сервер с наименьшим текущим временем ответа. Это позволяет направлять запросы к узлу, который, по всей видимости, находится в наилучшем состоянии для обработки данного запроса.
	
	\item Динамическая адаптация: Время ответа от узла может изменяться со временем в зависимости от нагрузки и состояния узлов. Метод наименьшего времени ответа учитывает эти изменения и позволяет системе адаптироваться к текущей ситуации.
	
	\item Предотвращение перегрузки: Этот метод также может включать в себя механизмы для предотвращения перегрузки узлов, например, не отправляя новые запросы на узел, который уже перегружен.
\end{itemize} 