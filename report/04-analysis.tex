\chapter{Аналитический раздел}
Алгоритмы балансировки можно разделить на статические и динамические \cite{uzel}.
\section{Динамическая балансировка}
Динамические алгоритмы осуществляют мониторинг состояния каждого
из узлов и выбирают «наилучший», в рассматриваемый момент времени, из них \cite{webmanage}. К динамическим алгоритмам относятся:
\begin{itemize}
	\item Least Connections
	\item Weighted Least Connections
	\item Least Time
\end{itemize}
\subsection{Least Connections}
Алгоритм Least Connections распределяет нагрузку между узлами в зависимости от количества активных соединений, обслуживаемых каждым узлом. Узел с наименьшим числом соединений будет обрабатывать следующий запрос, и узлы с большим числом соединений будет перераспределять свою нагрузку на узлы с меньшей загрузкой \cite{leastconnection}. 
% %Этот метод планирования включает в себя один из алгоритмов динамического планирования, так как он требует динамического вычисления активных соединений для каждого реального сервера.
%На~рисунке~\ref{img:leastconnections} проиллюстрирована работа метода наименьшего количества подключений  

% Этот метод планирования хорошо подходит для сглаживания распределения, когда поступает много запросов. Например, если есть два службы HTTP, такие как HTTP-1 (с 3 активными HTTP-транзакциями) и HTTP-2 (с 1 активной HTTP-транзакцией), то следующий запрос будет отправлен на службу HTTP-2, потому что HTTP-1 имеет больше активных транзакций, чем HTTP-2.

%\includeimage
%{leastconnections} % Имя файла без расширения (файл должен быть расположен в директории inc/img/)
%{f} % Обтекание (без обтекания)
%{h} % Положение рисунка (см. figure из пакета float)
%{\textwidth} % Ширина рисунка
%{Метод наименьшего количества подключений} % Подпись рисунка

\subsection{Weighted Least Connections}

Данный алгоритм комбинирует принципы алгоритмов Least Connections и Weighted Round Robin \cite{weightedroundrobin}. %\cite{mainsource}. 
Он учитывает как веса узлов, так и количество активных соединений. 
% Серверы с более высоким значением веса будут получать больший процент активных подключений в любой момент времени. 
%Вес сервера по умолчанию равен единице, и администратор IPVS или программа мониторинга могут назначить любой вес реальному серверу. 
Новое сетевое подключение предоставляется узлу, который имеет минимальное отношение количества текущих активных подключений к его весу \cite{mainsource}.

%В алгоритме планирования наименьшего взвешенного соединения (WLC) каждому серверу могут быть присвоены различные весовые коэффициенты производительности. Взвешенный алгоритм планирования наименьшего количества подключений делает с алгоритмом наименьшего количества подключений то же, что взвешенный алгоритм циклического перебора делает с алгоритмом циклического перебора. То есть, он вводит "вес", который основан на спецификациях каждого сервера.

\subsection{Least Time}
Алгоритм Least Time сочетает время отклика узла и активные соединения для определения лучшего узла \cite{balance}. 

Основные принципы метода наименьшего времени ответа включают:

\begin{itemize}
	\item Измерение времени ответа: Для эффективной работы этого метода необходимо непрерывно измерять время ответа от узла. Это может быть выполнено с помощью мониторинга, сбора статистики или других средств измерения производительности.
	
	\item Выбор узла с наименьшим временем ответа: Когда клиент отправляет запрос, система выбирает сервер с наименьшим текущим временем ответа. Это позволяет направлять запросы к узлу, который, по всей видимости, находится в наилучшем состоянии для обработки данного запроса.
	
	\item Динамическая адаптация: Время ответа от узла может изменяться со временем в зависимости от нагрузки и состояния узлов. Метод наименьшего времени ответа учитывает эти изменения и позволяет системе адаптироваться к текущей ситуации.
	
	\item Предотвращение перегрузки: Этот метод также может включать в себя механизмы для предотвращения перегрузки узлов, например, не отправляя новые запросы на узел, который уже перегружен.
\end{itemize} 