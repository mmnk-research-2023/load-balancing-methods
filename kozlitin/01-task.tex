ИУ7-54Б, 16\textunderscore\text{KOZ}, Козлитин

\chapter{Аналитическая часть}

\section{Weighted Response Time}

Weighted Response Time - алгоритм, при котором время ответа серверов определяет, какому серверу будет направлен следующий запрос.
Сервер, отвечающий на запрос быстрее всех, получает следующий запрос. \cite{sdnlb}

Пусть имеется $N$ запросов и $M$ узлов. Алгоритм состоит из следующих шагов.

\begin{enumerate}
    \item Cформировать массив $nodes$, содержащий узлы;
    \item Для каждого запроса из $N$:
    \begin{itemize}
        \item Установить переменную $target$ на первый доступный узел - $nodes[0]$.
	    \item Создать переменную $i = 1$.
        \item Пока $i < M$:
        \begin{itemize}
            \item если время ответа узла $target$ меньше времени ответа узла $nodes[i]$, то установить $target$ на текущий узел - $nodes[i]$.
		    \item увеличить значение $i$;
        \end{itemize}
        \item Отправить запрос на узел $target$.
	\end{itemize}
\end{enumerate}

\section{Random 2 (N) choices}

Random 2 (N) choices -  алгоритм, при котором определяется нагрузка 2 (N) серверов, выбранных случайным образом, и распределяется на наименее загруженный из них. 
В случае N=2 максимальная нагрузка на n серверов с высокой вероятностью составит \Theta(\log  \log n). \cite{pwr2choices}

Данный метод может быть использован, когда запрос требуется отправить на наименее загруженный сервер. 
Однако, полная информация о загрузке всех серверов может оказаться дорогостоящей для получения. 
Например, для получения загрузки на сервер может потребоваться отправка сообщения и ожидание ответа, 
обработка прерывания сервером. \cite{pwr2choices}

Альтернативный подход при котором информация о загрузке серверов не требуется, заключается в том, чтобы распределить запрос на слуайный сервер.
В таком случае максимальная нагрузка на n серверов с высокой вероятностью составит \Theta(\log n / \log \log n). \cite{pwr2choices}

Пусть имеется $K$ запросов и $M$ узлов, при этом $2 <= N <= M$. Алгоритм состоит из следующих шагов.

\begin{enumerate}
    \item Cформировать массив $nodes$, содержащий узлы;
    \item Для каждого запроса из $K$:
    \begin{itemize}
        \item Сфомировать массив $randoms$, содержащий $N$ узлов, выбранных случайным образом из массива $nodes$;
        \item Установить переменную $target$ на первый доступный узел - $randoms[0]$.
	    \item Создать переменную $i = 1$.
        \item Пока $i < M$:
        \begin{itemize}
            \item если нагрузка узла $randoms[i]$ меньше нагрузки узла $target$, то установить $target$ на узел $randoms[i]$.
		    \item увеличить значение $i$;
        \end{itemize}
        \item Отправить запрос на узел $target$.
	\end{itemize}
\end{enumerate}

\section{Resource based algorithm}

Resource based algorithm - алгоритм, при котором трафик распределяется балансировщиком нагрузки, в зависимости от текущей нагрузки на сервер. \cite{whatislb}

Специализированное программное обеспечение, называемое агентом, запускается на каждом сервере и рассчитывает использование ресурсов сервера, таких как его вычислительная мощность и память. 
Затем агент проверяется балансировщиком нагрузки на наличие достаточного количества свободных ресурсов перед распределением трафика на данный сервер. \cite{whatislb}

Пусть имеется $N$ запросов и $M$ узлов. Алгоритм состоит из следующих шагов.

\begin{enumerate}
    \item Cформировать массив $nodes$, содержащий узлы;
    \item Для каждого запроса из $N$:
    \begin{itemize}
        \item Cформировать массив $resources$, содержащий информацию об использовании ресурсов, соответствующим узлом;
        \item Установить переменную $target$ на первый доступный узел - $resources[0]$.
	    \item Создать переменную $i = 0$.
        \item Пока $i < M$:
        \begin{itemize}
            \item Узел $resources[i]$ обладает достаточным количеством свободных ресурсов для выполнения запроса:
            \begin{itemize}
                \item установить $target$ на узел $resources[i]$; 
                \item прекратить выполнение цикла;
            \end{itemize}
		    \item увеличить значение $i$;
        \end{itemize}
        \item Отправить запрос на узел $target$.
	\end{itemize}
\end{enumerate}